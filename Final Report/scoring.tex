Once the players have been determined, the next step is to predict how much each team will score.  
This is done by 1) approximating the number of scoring possessions each team will get; 2) removing the predicted number of scoring possessions lost due to turnovers; 3) using that number of scoring possessions and rebound probabilities to determine the number of scoring opportunities each team will receive.  
Those scoring opportunities are then divided among two point and three point attempts, with free-throw attempts calculated separately (due to the fact that free-throws can come from scoring attempts or non-shooting fouls if the team is in the bonus).  
These percentages were taken directly from the teams scouting report.  
Those shots are then distributed to the players most likely to shoot them, and the player's data is used to determine how many they will make.  
The distribution was determined by averaging the percentage of the team's two-point, three-point, and free throw shots that each player took over the season.  
Thus, if a certain player, on average, shot $20\%$ of the teams three-point shots, then $20\%$ of the number of three-point shots calculated were assigned to that player.  
Then say that that player had a $50\%$ average for three-point attempts.  
Then $10\%$ of the teams three-point shots are assigned as "made," contributing $3$ points to the team's total score each.  
This was done for all combinations of shot type and player.  
%This process will involve multiple machine learning algorithms, and Gibbs Sampling for determining player shot distributions.  
%Each shot will be randomly drawn from said distributions, with the game run many times to get an average winning percentage for each team.  
%This will give the probability that each team will win the game.  
After totaling the points from made baskets, the team with the most points is predicted to be the winner.  
Ties went to the team that was ranked with the higher seed in the tournament.  
During this process, all number of possessions, attempts, and made shots are rounded to integers to prevent illogical calculations (such as Nigel-Williams Goss of Gonzaga making 6.024 of his 7.593 three point shots).  