In conclusion, after organizing the data as described, standard classifiers such as logistic regression and support vector machines work just as well as traditional ranking methods for predicting March Madness.  
The scoring algorithm created is not much better than random chance, and currently not worth the time and energy to run it.  
While this is more than a little disappointing, it would makes sense; especially since there are other points of data that were not included in this model.  
Some of these variables that may play a significant role are distance the court is away from each teams respective homes, current winning streaks, schedule difficulty, rebounds, and others.  
The impact of these statistics should be considered in future iterations of the MYC classifier.  

The main success of this project has been the ability to predict the tournament with about as much accuracy as a ranking system, without ranking the teams.  
Since the vast majority of algorithms to predict the tournament ranks the teams, and then determined the winners based solely on those rankings, this provides a new and exciting chance to approach the problem from a different angle.  

This project has also opened up some new questions for further study.  
As stated before, one of these is how the best parameters change from year to year.  
Are they consistent, or arbitrarily random?  
Another good question is, what if you tried to classify teams based on how far they'll go in the tournament, as opposed to whether or not they'll win certain games?  
Would a method like that be beneficial, or would you end up with $20$ teams that could be in the elite eight and end up with less information than you started?  
Is previous success in the tournament a determining factor, or merely noise in the system?  
Are the best parameters for each classifier something that can, and should, be predicted, or random enough that it's impossible to guess which ones will perform best?  
The search for better ways to predict march madness certainly isn't over.  
As a result, the quest for the perfect bracket will continue.