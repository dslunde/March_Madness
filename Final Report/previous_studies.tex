Many previous algorithms have been used to try and solve this problem.  
The most common is a pagerank algorithm, similar to Google's algorithm for search results.  
Other methods worth mentioning are the Logistic Regression Markov Chain (LRMC) model, and the Massey Ratings, developed by Kenneth Massey, an assistant professor of mathematics at Carson-Newman University in Tennessee.  
These methods primarily focus on ranking teams based on a few factors, namely when the game was played, where the game was played, and what the score was.  
This project seeks to do things a little differently.  
Namely, by using team statistics (and a lot of them) to predict the winners of particular games directly, instead of ranking the teams first.  
While less common, the hope is that by using team statistics, the algorithm will be able to better identify potential nuances that contribute to cinderella teams (ones that get much further than predicted, such as Loyola-Chicago in 2018) and giant slayers (low ranked teams that beat a highly ranked team, like UMBC beating Virginia in 2018).  
While many attempts have been made to better understand and predict these events, no algorithm has consistently identified them.  